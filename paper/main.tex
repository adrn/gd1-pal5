\documentclass[modern]{aastex63}
% \documentclass[twocolumn]{aastex63}

\shorttitle{TODO}
\shortauthors{People et al.}

% Load common packages
\usepackage{microtype}  % ALWAYS!
\usepackage{amsmath}
\usepackage{amsfonts}
\usepackage{amssymb}
\usepackage{booktabs}

\usepackage{graphicx}
\usepackage{color}

\newcommand{\documentname}{\textsl{Article}}
\newcommand{\sectionname}{Section}
\renewcommand{\figurename}{Figure}
\newcommand{\equationname}{Equation}
\renewcommand{\tablename}{Table}

% Missions
\newcommand{\project}[1]{\textsl{#1}}

% Packages / projects / programming
\newcommand{\package}[1]{\textsl{#1}}
\newcommand{\acronym}[1]{{\small{#1}}}
\newcommand{\github}{\package{GitHub}}
\newcommand{\python}{\package{Python}}
\newcommand{\emcee}{\project{emcee}}

% For referee
\newcommand{\changes}[1]{{\color{red} #1}}

% Stats / probability
\newcommand{\given}{\,|\,}
\newcommand{\norm}{\mathcal{N}}

% Maths
\newcommand{\dd}{\mathrm{d}}
\newcommand{\transpose}[1]{{#1}^{\mathsf{T}}}
\newcommand{\inverse}[1]{{#1}^{-1}}
\newcommand{\argmin}{\operatornamewithlimits{argmin}}
\newcommand{\mean}[1]{\left< #1 \right>}
\newcommand{\mat}[1]{\mathbf{#1}}

% Unit shortcuts
\newcommand{\msun}{\ensuremath{\mathrm{M}_\odot}}
\newcommand{\kms}{\ensuremath{\mathrm{km}~\mathrm{s}^{-1}}}
\newcommand{\mps}{\ensuremath{\mathrm{m}~\mathrm{s}^{-1}}}
\newcommand{\pc}{\ensuremath{\mathrm{pc}}}
\newcommand{\kpc}{\ensuremath{\mathrm{kpc}}}
\newcommand{\kmskpc}{\ensuremath{\mathrm{km}~\mathrm{s}^{-1}~\mathrm{kpc}^{-1}}}
\newcommand{\masyr}{\ensuremath{\mathrm{mas}~\mathrm{yr}^{-1}}}

% Misc.
\newcommand{\bs}[1]{\boldsymbol{#1}}

% Astronomy
\newcommand{\DM}{{\rm DM}}
\newcommand{\feh}{\ensuremath{{[{\rm Fe}/{\rm H}]}}}
\newcommand{\df}{\acronym{DF}}
\newcommand{\logg}{\ensuremath{\log g}}
\newcommand{\Teff}{\ensuremath{T_{\textrm{eff}}}}
\newcommand{\FeH} {[\mathrm{Fe}/\mathrm{H}]}

% RR Lyrae
\newcommand{\rrab} {\mbox{RR\emph{ab}}}
\newcommand{\rrc} {\mbox{RR\emph{c}}}
\newcommand{\rrd} {\mbox{RR\emph{d}}}
\newcommand{\typeab} {\mbox{\emph{ab}}}
\newcommand{\typec} {\mbox{\emph{c}}}
\newcommand{\typedd} {\mbox{\emph{d}}}

% Gaia
\newcommand{\Gaia} {{\it Gaia}}
\newcommand{\DR}[1]{\acronym{DR#1}}
\newcommand{\G}{{\mathrm{G}}}
\newcommand{\Gbp}{G_{\mathrm{BP}}}
\newcommand{\Grp}{G_{\mathrm{RP}}}

% TO DO
\newcommand{\todo}[1]{{\color{red} TODO: #1}}

\graphicspath{paper/}

\begin{document}

\title{Constraining the Milky Way mass distribution with the GD-1 and Palomar 5 stellar streams}


\author[0000-0003-0872-7098]{Adrian~M.~Price-Whelan}
\affiliation{Center for Computational Astrophysics, Flatiron Institute,
             Simons Foundation, 162 Fifth Avenue, New York, NY 10010, USA}
\email{aprice-whelan@flatironinstitute.org}
\correspondingauthor{Adrian M. Price-Whelan}


\begin{abstract}
% Context
% Aims
% Methods
% Results
% Conclusions
\end{abstract}

\keywords{
Galaxy: halo ---
Galaxy: structure}

\section{Introduction} \label{sec:intro}

TODO


\section{Data} \label{sec:data}

\subsection{GD-1} \label{sec:data-gd1}

\begin{itemize}
  \item MS track from Gaia / PW2018
  \item RGB, BS, BHB from PS1 photometry
  \item Spectroscopy from BOSS/SDSS/SEGUE
  \item Hecto velocities from Ana?
\end{itemize}

\subsection{Pal 5} \label{sec:data-pal5}

\begin{itemize}
  \item MS track from legacy surveys / Bonaca2019
  \item RGB from Ibata2017
  \item RRL from PW2019
\end{itemize}


% \begin{figure}
% \begin{center}
% \includegraphics[width=\columnwidth]{ps1_completeness.pdf}
% \caption{PS1 RRL estimated completeness. The Pal 5 stream's track is shown with the solid line.}\label{fig:completeness_map}
% \end{center}
% \end{figure}


\acknowledgments
This work was performed in part during the Gaia19 workshop hosted by the Kavli Institute for Theoretical Physics at the University of California, Santa Barbara. 
The Flatiron Institute is supported by the Simons Foundation. 

This work has made use of data from the European Space Agency (ESA) mission
{\it Gaia} (\url{https://www.cosmos.esa.int/gaia}), processed by the {\it Gaia}
Data Processing and Analysis Consortium (DPAC,
\url{https://www.cosmos.esa.int/web/gaia/dpac/consortium}). Funding for the DPAC
has been provided by national institutions, in particular the institutions
participating in the {\it Gaia} Multilateral Agreement.

\software{
    Astropy \citep{astropy, astropy:2018},
    emcee \citep{emcee},
    gala \citep{gala},
    IPython \citep{ipython}
}

\bibliographystyle{aasjournal}
\bibliography{refs}

\end{document}
